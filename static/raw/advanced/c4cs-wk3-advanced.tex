\documentclass{article}
\usepackage[T1]{fontenc}

\usepackage{amssymb}
\usepackage{courier}
\usepackage{fancyhdr}
\usepackage{fancyvrb}
\usepackage[top=.75in, bottom=.75in, left=.75in,right=.75in]{geometry}
\usepackage{graphicx}
\usepackage{lastpage}
\usepackage{listings}
\lstset{basicstyle=\small\ttfamily}
\usepackage{mdframed}
\usepackage{parskip}
\usepackage{soul}
\usepackage{tabularx}
\usepackage{textcomp}
\usepackage{upquote}
\usepackage{xcolor}

% http://www.monperrus.net/martin/copy-pastable-ascii-characters-with-pdftex-pdflatex
\lstset{
upquote=true,
columns=fullflexible,
keepspaces=true,
literate={*}{{\char42}}1
         {-}{{\char45}}1
         {^}{{\char94}}1
}
\lstset{
  moredelim=**[is][\color{blue}\bf\small\ttfamily]{@}{@},
}

% http://tex.stackexchange.com/questions/40863/parskip-inserts-extra-space-after-floats-and-listings
\lstset{aboveskip=6pt plus 2pt minus 2pt, belowskip=-4pt plus 2pt minus 2pt}



\usepackage[colorlinks,urlcolor={blue}]{hyperref}
\usepackage[capitalise,nameinlink,noabbrev]{cleveref}
\crefname{section}{Question}{Questions}
\Crefname{section}{Question}{Questions}

\begin{document}


%%% TITLE INFO INSERTED HERE AUTOMATICALLY


\section{Pretty \texttt{PS1}}

% I use a complex color here to see if folks won't find simpler expressions
% (hopefully)
Open a new terminal and try the following commands in order:

\begin{minipage}[t]{0.5\textwidth}
\begin{lstlisting}[numbers=left]
echo -e "\\033[44;3;70;38;5;214m"
<hit enter again>
PS1="Hello World -- "
echo -e "\\033[44;3;70;38;5;214m"
pwd
ls
pwd
echo -e "\\033[44;3;70;38;5;214m"
ls --color=none
pwd
reset
\end{lstlisting}
\end{minipage}
\begin{minipage}[t]{0.5\textwidth}
  \begin{lstlisting}[numbers=left,firstnumber=12]
pwd
bash --norc
echo -e "\\033[44;3;70;38;5;214m"
ls
pwd
ls --color=auto
pwd
exit
PS1="\\033[44;3;70;38;5;214mHello Again -- "
ls
\end{lstlisting}
\end{minipage}

\noindent
What happened to your terminal as you ran these commands?
Play around with some other forms of \texttt{PS1} and other colors, see what
happens. Why does calling \texttt{ls} sometimes reset things?

\medskip
\noindent
Create a custom \texttt{PS1} for yourself. Look into some of the options for
\texttt{PS1}, you will need to explain why you added the options you did and
decided against options you didn't choose.

\medskip
\noindent
Extend your PS1 by writing a bash function the changes your prompt in a way
that is not built-in to bash. Some examples: Add an asterisk if your are in a
git repository with uncommitted changes. Change the color if you are currently
in a shared directory (i.e. in a Dropbox folder). Change the color if the
current directory will not be saved across a reboot (i.e. if you are you are
somewhere in the \texttt{/tmp} directory).

\subsection*{Submission checkoff:}
\begin{itemize}
  \item[$\square$] Explain what \texttt{PS1} does
  \item[$\square$] Explain what you did to customize \texttt{PS1} and
    \textbf{why} you chose the customizations that you did.
    \begin{itemize}
      \item[$\square$] Explain how your custom function works.
    \end{itemize}
  \item[$\square$] Explain what the \texttt{PS2} variable controls. Change
    \texttt{PS2} from the default and show an example.
  \item[$\square$] Type \texttt{set -x}. Then type \texttt{ls}. Explain
    all of the output.
\end{itemize}


\section{Understanding tab completions}

\medskip
\noindent
Open a new terminal and try typing the following (note $<$tab$>$ means press
the tab key):

\begin{lstlisting}[numbers=left]
p<tab><tab>
y
<ctrl-c>
pi<tab><tab>
pin<tab><tab>
ping<tab><tab>
ping <tab><tab>
PATH=<enter>
p<tab><tab>
\end{lstlisting}

\medskip
\noindent
In addition to finding programs, tab completions can help you to use a program
correctly by hinting at what arguments a program accepts, try this:

\begin{lstlisting}[numbers=left,firstnumber=10]
# Open a new terminal (or manually set your PATH correctly again)
ping <tab><tab>
ping -<tab><tab>
ping -I<tab><tab>
ping -Q<tab><tab>
ping -Q 0 <tab><tab>
\end{lstlisting}

\noindent
Today, most programs include tab completion support, but this is a remarkably
manual process. Check out the contents of the
\texttt{/usr/share/bash-completion/completions/} directory.

\medskip
\noindent
Now take a look at \texttt{/usr/share/bash-completion/completions/ping}.
There's a lot going on in this example, but try to see if you can understand
some of how the completions are working. What do you think the result of
\texttt{ping~-T~$<$tab$><$tab$>$} will be?

\medskip
\noindent
(Hint: \texttt{||} in bash means ``if the previous thing failed, do the next
thing''. What's the output of \texttt{echo \$OSTYPE}? Will that equality pass
or fail?)


\subsection*{Submission checkoff:}
\begin{itemize}
  \item[$\square$] Why is the list of tab completions different between lines
    1 and 9?
  \item[$\square$] What is the default tab completion behavior for a program
    if no custom completion function has been written?
\end{itemize}


\pagebreak


\section{Testing Made Easier}

\medskip
\noindent
When working on a project in EECS, you should be writing test cases to make
sure your program is functioning properly. Checking your test cases can be
tedious, but fortunately for us, scripting can help make it easier to run your
test cases and report which ones are passing and which ones are failing. Here,
we will do just that.

\noindent
Go ahead and download some files from this URL, extract them, and take a look.
You should see 3 files. \texttt{testPass.sh}, \texttt{testFail.sh}, and
\texttt{testTimeout.sh}. These files will pass/fail according to their names.
You should not need to modify their contents.

%%% TODO: Change semester
\begin{Verbatim}[fontsize=\footnotesize]
  > wget https://c4cs.github.io/static/f17/advanced/wk3-advanced-p3-files.tar.gz
  > # eXtract Ze Files
  > tar xzf wk3-advanced-p3-files.tar.gz
\end{Verbatim}

\noindent
Write a shell script which takes all files in the current directory with
``test'' somewhere in the name, runs each of them, and reports whether the
program passed or failed the test case by printing ``PASSED'', ``FAILED''. Your
script should also stop programs from running for more than 3 seconds and
print ``TIMEOUT''.

An example test runner file to use as a framework is below. A solution may follow the
structure very closely, but remember, there are almost always multiple ways of
solving the same problem, so feel free to deviate from the suggestions.

One requirement is to not hard code the list of filenames into the script, this
is where shell globbing/wildcards can help (we suggest looking these up to learn
more). This requirement is so that you can run your master script in any
directory with files including ``test'' in the name and it should
Just Work \textsuperscript{TM}.


\begin{minipage}[t]{0.5\textwidth}
\begin{lstlisting}[numbers=left]
#!/bin/bash

# loop through all files with `test` in the name
# (learning more about for loops and shell globbing/wildcards will help for this)

    # for each file, execute it
    # (you may need to execute the file with another program that will handle the
    # timeout case)

    # save the return code of the previous execution into a variable

    # check the return code for timeout, print "TIMEOUT" if so

    # check the return code for failure, print "FAILURE" if so

    # check the return code for success, print "SUCCESS" if so

\end{lstlisting}
\end{minipage}

\subsection*{Submission checkoff:}
\begin{itemize}
  \item[$\square$] How does the \texttt{exit} command work? (Hint: How is this similar to
    \texttt{return} in C/C++ ?)
  \item[$\square$] How can a test case report to our test running script whether the target program passed or failed?
  \item[$\square$] How can you tell if a program is running too long?
  \item[$\square$] Can you demo your code?
\end{itemize}



\end{document}
