\documentclass{article}
\usepackage{amssymb}
\usepackage{comment}
\usepackage{courier}
\usepackage{fancyhdr}
\usepackage{fancyvrb}
\usepackage[T1]{fontenc}
\usepackage[top=.75in, bottom=.75in, left=.75in,right=.75in]{geometry}
\usepackage{graphicx}
\usepackage{lastpage}
\usepackage{listings}
\lstset{basicstyle=\small\ttfamily}
\usepackage{mdframed}
\usepackage{parskip}
\usepackage{ragged2e}
\usepackage{soul}
\usepackage{upquote}
\usepackage{xcolor}

% http://www.monperrus.net/martin/copy-pastable-ascii-characters-with-pdftex-pdflatex
\lstset{
upquote=true,
columns=fullflexible,
literate={*}{{\char42}}1
         {-}{{\char45}}1
         {^}{{\char94}}1
}
\lstset{
  moredelim=**[is][\color{blue}\bf\small\ttfamily]{@}{@},
}

% http://tex.stackexchange.com/questions/40863/parskip-inserts-extra-space-after-floats-and-listings
\lstset{aboveskip=6pt plus 2pt minus 2pt, belowskip=-4pt plus 2pt minus 2pt}

\usepackage[colorlinks,urlcolor={blue}]{hyperref}

\begin{document}

\fancyfoot[L]{\color{gray} C4CS -- W'16}
\fancyfoot[R]{\color{gray} Revision 1.0}
\fancyfoot[C]{\color{gray} \thepage~/~\pageref*{LastPage}}
\pagestyle{fancyplain}


\title{\textbf{Advanced Homework 10\\}}
\author{Assigned: Friday, March 18th}
\date{\textbf{\color{red}{Due: Before the first lecture on Friday, April 1}}}
\maketitle


\section*{Submission Instructions}
To receive credit for this assignment you will need to stop by someone's
office hours, demo your running code, and answer all the bolded questions.

\section*{From \texttt{gprof} to \texttt{gcov}}

gcov is a more advanced profiler. While gprof gave only function-level
information, gcov can give line-by-line statistics, letting you find hot loops
or other fine-grained issues.

\textbf{What extra flags do you have to give the compiler to get gcov working?}

\textbf{The \texttt{-pg} flags added calls to \texttt{\_mprofil} to every
  function, roughly, what do these new flags do?}
\vspace{1cm}

One interesting thing that gcov can show you is ``taken/not~taken'' statistics
for branches (how often was this if statement true).

Modify the code from the homework to include some branches. Include some that
are always taken, some that are never taken, and some that are taken
probabilistically with 25, 50, and 75\% chance (\texttt{man~3~rand} will be
helpful). The probability that each branch is taken should be read from a
file or command-line argument, i.e.\ you must be able to change the
probability that a branch will be taken \emph{without} recompiling your code.
Your branches should run many, many times.

\textbf{Show the output of gcov's branch metrics and show that your
  probabilities are working as expected.}

\textbf{Show how changing the probabilities leads to different results.}
\vspace{1cm}

Gcc supports something called \emph{Profile Guided Optimization}. The idea
here is that gcc can use the result of a profiling run to guide it
compilation. For example, if a branch is likely to be taken, it can provide
hints to the CPU that that branch will be taken. In some cases, it will even
emit code that removes the branch, assumes it was taken, and checks at the end
whether it was right, undo-ing the code that was run if it was wrong.

\textbf{Show how to add coverage information to compilation and optimization.}

\textbf{Can you find a program that runs faster with coverage information? Any
  previous projects that do searches are good candidates. Anything
  long-running will be better.}
If you can't find something that runs noticeably faster, can you at least show
an example where the compiled output (\texttt{objdump}) changes? Can you give
a \emph{rough} explanation of what changed (you \emph{do not} need to deeply
understand x86 assembly! An educated guess is fine). objdump gives a lot of
information, you only need to look at specific functions though -- use search.

\end{document}
