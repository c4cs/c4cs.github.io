\documentclass{article}
\usepackage{amssymb}
\usepackage{comment}
\usepackage{courier}
\usepackage{fancyhdr}
\usepackage{fancyvrb}
\usepackage[T1]{fontenc}
\usepackage[top=.75in, bottom=.75in, left=.75in,right=.75in]{geometry}
\usepackage{graphicx}
\usepackage{lastpage}
\usepackage{listings}
\lstset{basicstyle=\small\ttfamily}
\usepackage{mdframed}
\usepackage{parskip}
\usepackage{ragged2e}
\usepackage{soul}
\usepackage{upquote}
\usepackage{xcolor}

% http://www.monperrus.net/martin/copy-pastable-ascii-characters-with-pdftex-pdflatex
\lstset{
upquote=true,
columns=fullflexible,
literate={*}{{\char42}}1
         {-}{{\char45}}1
         {^}{{\char94}}1
}
\lstset{
  moredelim=**[is][\color{blue}\bf\small\ttfamily]{@}{@},
}

% http://tex.stackexchange.com/questions/40863/parskip-inserts-extra-space-after-floats-and-listings
\lstset{aboveskip=6pt plus 2pt minus 2pt, belowskip=-4pt plus 2pt minus 2pt}

\usepackage[colorlinks,urlcolor={blue}]{hyperref}

\begin{document}

\fancyfoot[L]{\color{gray} C4CS -- W'16}
\fancyfoot[R]{\color{gray} Revision 1.0}
\fancyfoot[C]{\color{gray} \thepage~/~\pageref*{LastPage}}
\pagestyle{fancyplain}


\title{\textbf{Advanced Homework 9\\}}
\author{Assigned: Friday, March 11th}
\date{\textbf{\color{red}{Due: Before the first lecture on Friday, March 25}}}
\maketitle


%\section*{Submission Instructions}

\section*{Building a Command Reference}

A few weeks ago, we collected all of the commands that we have seen in this
class so far. Now we're going to take that list build up a reference resource
for current and future students. We've gotten things started at
\href{https://c4cs.github.io/reference}{c4cs.github.io/reference}.

The idea here is that you are best qualified to generate examples that are
accessible to you and useful for you.

\subsection*{GitHub Pages}

Notice how the course homepage is \texttt{c4cs.\ul{github.io}}? GitHub has a
really neat feature called \href{https://pages.github.com/}{GitHub Pages},
that will turn a repository into a website -- hosted for free.\footnote{
  My \href{https://github.com/ppannuto/patpannuto.com}{personal website} is also
  \href{https://github.com/ppannuto/ppannuto.github.io}{hosted this way},
  though it does not use jekyll.
} What's really nice about this is that it means it is very easy for many
people to collaborate to develop a website, it's just a repository!

In the simplest setup, GitHub will simply serve the files in the repository as
static web pages. Writing lots of HTML by hand, however, can be a pain, so
GitHub supports a \emph{site generator}, \href{https://jekyllrb.com/}{jekyll}.
Using jekyll, adding an update to the homepage is as simple as adding
\href{https://github.com/c4cs/c4cs.github.io/blob/master/updates/_posts/2016-02-27-spring-break-advanced.md}{a text file}.
The site will also
\href{https://github.com/c4cs/c4cs.github.io/blob/master/index.html#L32}%
{automatically hide updates older than one month}.

\subsection*{Getting Going}

First, install ruby: \texttt{sudo apt-get install ruby ruby-dev zlib1g-dev}

Then, install \href{http://bundler.io}{bundler}: \texttt{sudo gem install bundler}

Now grab a copy of course website repository
(\url{https://github.com/c4cs/c4cs.github.io}), and follow the directions
in the readme.

Once things are running, it should print out
\begin{quote}\tt
  Server address: http://127.0.0.1:4000/
\end{quote}

Visit \url{http://127.0.0.1:4000/} in your browser and you should see your own
local copy of the course website! Try making some changes to the site and then
refresh the page in your browser.

\subsection*{The Assignment:}

Choose one of the commands that have not yet been documented and add
documentation for it. Alternatively, pick a command that isn't on the
list and add documentation for it.
Check out \texttt{ls} as a good, complete example.

Once you have made your changes, issue a
\href{https://guides.github.com/activities/contributing-to-open-source/#contributing}{Pull~Request}.

We'll look over your changes and give you feedback. Once we merge your
changes, this assignment is complete!


\end{document}
