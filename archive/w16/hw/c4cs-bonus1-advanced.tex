\documentclass{article}
\usepackage{amssymb}
\usepackage{courier}
\usepackage{fancyhdr}
\usepackage[top=.75in, bottom=.75in, left=.75in,right=.75in]{geometry}
\usepackage{lastpage}
\usepackage{listings}
\lstset{basicstyle=\small\ttfamily}
\usepackage{soul}
\usepackage{xcolor}

\usepackage[colorlinks,urlcolor={blue}]{hyperref}

\begin{document}

\fancyfoot[L]{\color{gray} C4CS -- W'16}
\fancyfoot[R]{\color{gray} Revision 1.0}
\fancyfoot[C]{\color{gray} \thepage~/~\pageref*{LastPage}}
\pagestyle{fancyplain}


\title{\textbf{Bonus Advanced Homework\\Upstreaming}}
\author{Released: Friday, January 22, 11:00AM}
\date{
\textbf{\color{red}{Milestone~1 (Must start by) Friday, March 18}}\\
~\\
\textbf{\color{red}{Due: Before lecture on Friday, April 8}}
}
\maketitle


\section*{Submission Instructions}
This assignment is different than most. You have nearly 10~weeks, however you
will have to start early because \emph{you do not control the timetable}.

\medskip
\noindent
To receive credit for this assignment you will need to stop by someone's
office hours, explain your improvement, and demonstrate your interaction with
an open source project.

\section{Upstreaming Code}

One of the coolest parts of computer science is that with only a small amount
of learning and experience, you are qualified to provide real contributions
to software projects used by hundreds to millions of people around the
world.\footnote{%
  Don't believe me? Go check out
  \href{http://write.flossmanuals.net/gsocstudentguide/am-i-good-enough/}{Google
  Summer of Code's ``Am I Good Enough?''}. The summer of code program can be
  an excellent way to spend a summer in lieu of a more traditional internship.
  \href{https://developers.google.com/open-source/gsoc/timeline}
  {Applications open February~29 and close March~25 this year}.
}
There is a lot of code to be written in the world and not nearly enough people
qualified to write it.

Nearly everyone in the world uses open source software every day. As a
computer scientist you will very likely use a lot more open source software
than most people.
Contributing back to an open source project is a great way to pay back into
this ecosystem, a \emph{great} way to gain some real-world software
experience, and a nice feather for any r\'esum\'e.\footnote{%
  Though for many reasons, it should not \emph{replace} your r\'esum\'e.
  \href{https://blog.jcoglan.com/2013/11/15/why-github-is-not-your-cv/}
  {This post} is a long read on the subject (it's also worth reading several
  of the linked posts). This area is interesting reading for understanding
  some of how computer science is evolving as a field culturally.
}


\subsection*{How to Get Started}

Getting started is the hardest part. There is an overwhelming number of
projects to choose from.
Big projects, such as Mozilla, often have
\href{https://developer.mozilla.org/en-US/docs/Introduction}
{documentation explaining how to contribute},
their bug trackers will track and let you query things like
\href{https://bugzilla.mozilla.org/buglist.cgi?resolution=---&classification=Client%20Software&emailtype1=regexp&status_whiteboard_type=allwordssubstr&query_format=advanced&emailassigned_to1=1&status_whiteboard=%5Bgood%20first%20bug%5D&email1=nobody}
{[good first bug]},
and their contributors are often friendly to beginners, with
\href{https://bugzilla.mozilla.org/show_bug.cgi?id=361983#c11}
{good advice for getting started}.
The negative with big projects is that they are big. You have to learn a large
codebase in order to get started at all.

On the flip side, small projects, such as
\href{https://github.com/kylelady/omnomnorth}{omnomnorth}
can be easier to make changes to, but harder to get started with as they do
not have mature ``getting started'' guides and their developer(s) (if there's
even more than one) may not have the time (or patience) to mentor young
coders.

If possible, I recommend picking something that you use. It's easier to
motivate yourself to work on something you use, and there's something pretty
cool about \href{https://github.com/um-cseg/chez-betty/}{using your own
software every day}.
\href{https://github.com/Aluxian/Facebook-Messenger-Desktop}{Many}
\href{https://github.com/GNOME/gnome-terminal}{things}
\href{https://github.com/gnachman/iTerm2}{you}
\href{https://github.com/tmux/tmux}{(should)}
\href{https://trac.transmissionbt.com/wiki/Building}{use}
\href{https://github.com/scummvm/scummvm}{are}
\href{https://github.com/GNOME/gimp}{open}
\href{https://github.com/videolan/vlc}{source}.
Many libraries, such as the
\href{https://bitbucket.org/birkenfeld/pygments-main}
{near-universal syntax highlighting library}, the
\href{https://github.com/jgm/pandoc}
{near-universal document converter}, or a
\href{https://github.com/tartley/colorama}
{simple library for pretty terminal output}
have a lot of low-hanging fruit (check out the ``Issues'' tab).

\newpage
\subsection*{The Assignment:}

Pick an active open source project and make a meaningful contribution to it.
%
\begin{quote}
  \textbf{\emph{active}, adj.} A project used by a reasonable number (say
  100+) of people that you don't know.
\end{quote}
%
\begin{quote}
  \textbf{\emph{meaningful}, adj.} A non-trivial change. More than a one-word
  documentation fix. Bigger documentation fixes, such as adding a thorough
  installation / getting started guide to a project that doesn't have one is a
  great contribution. Fixing an open issue/bug is great. Adding a feature is
  great. Anything 10 lines+ of code is likely good. A one character change
  that fixes a reported bug is also good, however.

  Ultimately, this is
  \href{https://en.wikipedia.org/wiki/I_know_it_when_I_see_it}{subjective}.
  However, if you feel you can justify to a course staff that it was
  meaningful, then it probably was.
\end{quote}
%
Having you contribution accepted, especially your first one, often takes some
back-and-forth. The project owner will want you to make small changes. Be
prepared for this, it doesn't mean that your overall patch is bad! This does
mean, however, that it can take a little time for your patch to be accepted.

\medskip
\noindent
Submitting a patch to an open source project is a public process. Usually via
something like a pull request if the project uses a collaboration site like
GitHub or BitBucket, or by sending an e-mail to a public mailing list. To
ensure you start early enough to have a decent chance of your patch being
merged, Milestone~1 is the deadline for your first public action. That is,
\hl{you must start the public process of submitting your contribution(s)
by March~18th}.

\medskip
\noindent
If you have previously contributed to an open source project, consider this
assignment a good excuse to go contribute again. Contributions made prior to
this semester don't count.

\subsection*{Submission checkoff:}
\begin{itemize}
  \item[$\square$] What project you picked, why
  \item[$\square$] What problem/feature you picked, why
  \item[$\square$] How did you implement/fix it?
  \item[$\square$] Show a history of your communication with the upstream maintainers
  \item[$\square$] Show that your patch was accepted -- or (judgment call)
    that you put forth a solid effort thus far and will likely continue to
    put forth the effort to get this patch in
\end{itemize}

%\newpage
%\section*{Appendix}
%
%Some common courtesy things:
%
%\begin{itemize}
%  \item \textbf{DO NOT} change whitespace or other formatting
%  \item \textbf{DO} adhere to the project's existing conventions -- if it's a
%    Python project that uses tab characters instead of 4~spaces, go with it;
%    if it's a C~project that calls functions as \texttt{add~(a, b)} instead of
%    \texttt{add(a, b)}\footnote{
%      Some coding conventions do this so that it is easy to search for the
%      function's declaration, \texttt{int add(int a, int b)} (no space), and
%      callers of the function, \texttt{add~(a, b)} (space), without having to
%      know anything else about the fuction (i.e. serach for \texttt{"add("})
%    } then be consistent with that too.
%\end{itemize}

\end{document}
